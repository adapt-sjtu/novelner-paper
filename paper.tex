\documentclass[11pt,letterpaper]{article}
\usepackage{emnlp2017}
\usepackage{times}
\usepackage{latexsym}

\usepackage{etoolbox}
\newcommand{\zerodisplayskips}{%
	\setlength{\abovedisplayskip}{0pt}%
	\setlength{\belowdisplayskip}{0pt}%
	\setlength{\abovedisplayshortskip}{0pt}%
	\setlength{\belowdisplayshortskip}{0pt}}
\appto{\normalsize}{\zerodisplayskips}
\appto{\small}{\zerodisplayskips}
\appto{\footnotesize}{\zerodisplayskips}
\usepackage{enumitem}
\setlist{nosep}

\usepackage{amsmath,amsfonts,amsthm}
\usepackage{color}
\usepackage{url}
\usepackage{array}
\usepackage{epsfig}
\usepackage{graphicx}
\usepackage{caption}
\usepackage{subcaption}
\usepackage{array}
\usepackage{ragged2e}
\usepackage{pbox}


\usepackage[ruled,vlined,boxed,linesnumbered]{algorithm2e}
\SetAlFnt{\small}
\SetAlCapFnt{\small}
\SetAlCapNameFnt{\small}
\usepackage{float}
\restylefloat{table}
\newcolumntype{P}[1]{>{\RaggedRight\hspace{0pt}}p{#1}}
\newcolumntype{L}[1]{>{\raggedright\let\newline\\\arraybackslash\hspace{0pt}}m{#1}}
\newcolumntype{C}[1]{>{\centering\let\newline\\\arraybackslash\hspace{0pt}}m{#1}}
\newcolumntype{R}[1]{>{\raggedleft\let\newline\\\arraybackslash\hspace{0pt}}m{#1}}


\newcommand{\secref}[1]{Section \ref{#1}}
\newcommand{\figref}[1]{Figure \ref{#1}}
\newcommand{\eqnref}[1]{Eq. (\ref{#1})}
\newcommand{\tabref}[1]{Table \ref{#1}}
\newcommand{\exref}[1]{Example \ref{#1}}
\newcommand{\algref}[1]{Algorithm \ref{#1}}
\newcommand{\socvec}{SocVec}
\newcommand{\argmin}{\operatornamewithlimits{argmin}}
\newcommand{\argmax}{\operatornamewithlimits{argmax}}
\newtheorem{example}{Example}
\newtheorem{lemma}{Lemma}
\newtheorem{definition}{Definition}
\newcommand{\cut}[1]{}

\newcommand{\li}{\uline{\hspace{0.5em}}}
\newcommand{\BL}[1]{\textcolor{blue}{Bill: #1}}
\newcommand{\HY}[1]{\textcolor{red}{Jessie: #1}}
\newcommand{\KZ}[1]{\textcolor{green}{Kenny: #1}}
\newcommand{\SH}[1]{\textcolor{green}{Seung: #1}}
\newcommand{\FX}[1]{\textcolor{blue}{Frank: #1}}

% Uncomment this line for the final submission:
%\emnlpfinalcopy

%  Enter the EMNLP Paper ID here:
\def\emnlppaperid{}

% To expand the titlebox for more authors, uncomment
% below and set accordingly.
% \addtolength\titlebox{.5in}    

\newcommand\BibTeX{B{\sc ib}\TeX}

\title{Multi-channel BiLSTM-CRF Model for\\  Emerging Named Entity Recognition in Social Media  }
\author{}

\begin{document}
\maketitle
\begin{abstract}
In this paper, we present our multi-channel neural architecture for recognizing emerging named entity in social media messages,
which we applied in the \textit{Novel and Emerging Named Entity Recognition} shared task at the EMNLP
2017 Workshop on Noisy User-generated Text (W-NUT). 
The major difficulties in this shared task are the short and noisy nature of user-generated text as well as the unseen named entities with novel surface forms. 

We propose a novel approach, which incorporates comprehensive word representations with multi-channel information and Conditional Random Fields (CRF) into a traditional Bidirectional Long Short-Term Memory
(BiLSTM) neural network
% to automatically learn orthographic features 
without using any additional hand-craft features such as gazetteers. 
In comparison with other systems participating in the shared task, our system won the 2nd place.
%\footnote{Specifically, we won the second place on the ``surface form F1'' track, the third place on the ``entity  F1''  track.} on the overall average performance.
\footnote{Our code can be achieved at \url{https://github.com/adapt-sjtu/novelner}  (The code will be further organized and described in the meantime when this report is under review。)}
\end{abstract}
\section{Introduction}
\label{sec:intro}
Named entity recognition (NER) is one of the first and most important steps in Information Extraction pipelines. 
Generally, it is to identify mentions of entities (persons, locations, organizations, etc.)
within unstructured text. 
However, the diverse and noisy nature of user-generated content as well as the emerging entities with novel surface forms make NER in social media messages more challenging.

The first challenge brought by user-generated content
%such as Twitter, Reddit and YouTube, 
is its unique characteristics: short, noisy and informal. 
For instance, tweets are typically short since the number of characters is restricted to 140 and people indeed tend to pose short messages even in social media without such restrictions, such as YouTube comments and Reddit.
\footnote{The average length of the sentences in this shared task is about 20 tokens per sentence.}
Hence, the contextual information in a sentence is very limited.
Apart from that, 
the use of colloquial language makes it more difficult for existing NER approaches to be reused, which mainly focus on a general domain and formal text ~\cite{baldwin2015shared, derczynski2015analysis}. 
%State-of-the-art NER softwares (e.g. Standford Corenlp) are less effective on such social media messages~\cite{derczynski2015analysis}.
%Due to the informal and contemporary nature of these micro-posts, performance still lags far behind that on formal text genres such as newswire. 

What makes Twitter NER more challenging is the fact that there are large amounts of rare and emerging named entities among the text.
This shared task focuses on identifying unusual, previously-unseen entities in the context of emerging discussions. Named entities form the basis of many modern approaches to other tasks (like event clustering and summarization), but recall on them is a real problem in noisy text - even among annotators. This drop tends to be due to novel entities and surface forms. Take for example the tweet ``so.. \textit{kktny} in 30 mins?'' - even human experts find entity \textit{kktny} hard to detect and resolve. It actually refers to a new TV series called ``\textit{Kourtney and Kim Take New York}''. This task will evaluate the ability to detect and classify novel, emerging, singleton named entities in noisy text.

Detecting commonly-mentioned entities tends to be easier than the rarer, more unusual surface forms. Similarly, entities with unusual surface forms, or that are simply rare, tend to be tougher to detect~\cite{augenstein2017generalisation}, with recall being a significant problem in rapidly-changing text types~\cite{derczynski2015analysis}. However, the entities that are common in newly-emerging texts such as newswire or social media are often new, not having been mentioned in prior datasets. This poses a challenge to NER systems, where in many deployments, unusual, previously-unseen entities need to be detected reliably and with high recall. In the shared task, we are provided with turbulent data containing few repeated entities, drawn from rapidly-changing text types or sources of non-mainstream entities.

The goal of this paper is to present a novel Multi-channel BiLSTM-CRF neural network approach for Twitter NER. The system is given the emphasize on the rare and emerging named entities that have few occurrences in the Twitter corpus. The system is to identify the span of named entities from given tweet texts, and classify the type of the NEs into 6 categories, namely Person, Location (including GPE, facility), Corporation, Consumer good (tangible goods, or well-defined services), Creative work (song, movie, book, and so on) and Group (subsuming music band, sports team, and non-corporate organizations).



\section{Problem Definition}
The NER is a classic sequence labeling problem, in which we are given a sentence, in the form of a sequence of tokens $\mathbf{w} = (w_1, w_2, ..., w_n)$, and we are required to output a sequence of token labels $\mathbf{y} = (y_1, y_2, ..., y_n)$. In this specific task, we use the  standard BIO2 annotation, and each named entity chunk are classified into 6 categories, namely Person, Location (including GPE, facility), Corporation, Consumer good (tangible goods, or well-defined services), Creative work (song, movie, book, and so on) and Group (subsuming music band, sports team, and non-corporate organizations).
\section{Approach}
\label{sec:approach}
In this section, we will first introduce the overview of our proposed model and then present each part of the model in detail.

\subsection{Overview}
\figref{fig:overall} shows the overall structure of our proposed model, 
instead of solely using the original pretrained word embeddings as the final word representations, 
we construct a comprehensive word representation for each word in the input sentence.
This comprehensive word representations contain the character-level sub-word information, the original pretrained word embeddings and multiple syntactical features. 
Then, we feed them into a Bidirectional LSTM layer, and thus we have a hidden state for each word. 
The hidden states are considered as the feature vectors of the words by the final CRF layer, from which we can decode the final predicted tag sequence for the input sentence.

\subsection{Comprehensive Word Representations}
In this subsection, we present our proposed comprehensive word representations. 
We first build character-level word representations from the embeddings of every characters in each word using a bidirectional LSTM. 
Then we further incorporate the final word representation with the embedding of the syntactical information of each token, such as the part-of-speech tag, the dependency role, the position in the sentence and the position of the head. \BL{i am not sure about the HEAD. maybe we can find a better description.}
Finally, we combine the original word embeddings with the above two parts to obtain the final comprehensive word representations.

\subsubsection{Character-level Word Representations}
In noisy user-generated text analysis, sub-word (character-level) information is much more important than that in normal text analysis for two main reasons:
1) People are more likely to use novel abbreviations and morphs to mention entities, which are often out of vocabulary and only occur a few times. 
Thus, solely using the original word-level word embedding as features to represent  words is not adequate to capture the characteristics of such mentions.
2) Another reason why we have to pay more attention to character-level word representation for noisy text is that it is can capture the orthographic or morphological information of both formal words and Internet slang.
\footnote{\BL{the abbr. examples and morph stuff should be mentioned in the introduction part as one of the reasons why twitter NER is hard and cite Heng Ji's paper}} 

There are two main network structures to make use of character embeddings: 
one is CNN (Ma and Hovy, 2016 \cite{} ) and the other is BiLSTM(Lample et al. 2016).
BiLSTM turns to be better in our experiment on development dataset and we will discuss the reason in \secref{sec:eval}.
Thus, we follow Lample et al. (2016) to build a BiLSTM network to encode the characters in each token as \figref{arg1} shows. \BL{see Figure 4 in Lample's paper.}
We finally concatenate the forward embedding and backward embedding to the final character-level word representation.
%such as ``kktny'' meaning \textit{Kourtney and Kim Take New York}, ``himym'' meaning ``How I Met Your Mother''. 
%Also, people create new and emerging morphs for entities for fun or avoiding censorship, which a


%http://ac.els-cdn.com/S1877050917306580/1-s2.0-S1877050917306580-main.pdf?_tid=139e4e02-617c-11e7-9a7c-00000aab0f27&acdnat=1499257299_dd0b70d56f16720625b3ac5eec4d7284

\subsubsection{Syntactical Word Representations}
We argue that the syntactical information, such as POS tag and dependency role, should also be explicitly considered as context features of each token in the sentence. 
\BL{Jessie plz introduce how we extract the twitter pos tags etc. from the Tweebo parser. Better show an example }.
 
\subsubsection{Combination with Word-level Word Representations}
After obtaining the above two additional word representations, we combine them with the original word-level word representations, which are just traditional word embeddings. 
We utilize pretrained word embeddings from glove\footnote{\url{http://nlp.stanford.edu/data/glove.twitter.27B.zip}}.
For all out-of-vocabulary words, we assign their embeddings by sampling from range $\left[-\sqrt{\frac{3}{\text{dim}}}, +\sqrt{\frac{3}{\text{dim}}}\right]$, where \textit{dim} is the dimension of word embeddings.

To sum up, our comprehensive word representations are the concatenation of three parts: 1) character-level word representations, 2) syntactical word representation and 3)  original pretrained word embeddings.

\subsection{BiLSTM Layer}
LSTM based networks are proven to be effective in sequence labeling problem for they have access to both past and the future contexts. 
Whereas, hidden states in unidirectional LSTMs only takes information from the past, which may be adequate to classify the sentiment  is a shortcoming for labeling each token.
Bidirectional LSTMs enable the hidden states to capture both historical and future context information and then to label a token.

Mathematically, the input of this BiLSTM layer is a sequence of comprehensive word representations (vectors) for the tokens of the input sentence,  denoted as 
$( \mathbf{x_1}, \mathbf{x_2},...,\mathbf{x_n})$. 
The output of this BiLSTM layer is a sequence of the hidden states for each input word vectors, denoted as 
$( \mathbf{h_1}, \mathbf{h_2},...,\mathbf{h_n})$. 
Each final hidden state is the concatenation of the forward $\overleftarrow{\mathbf{h_i}}$ and backward $\overrightarrow{\mathbf{h_i}}$ hidden states.
We know that 

$$\overleftarrow{\mathbf{h_i}}= \text{lstm}(\mathbf{x_i}, \overleftarrow{\mathbf{h_{i-1}}})~\text{,}~\overrightarrow{\mathbf{h_i}}= \text{lstm}(\mathbf{x_i}, \overrightarrow{\mathbf{h_{i+1}}})$$ 

$$\mathbf{h_i} = \left[ \overleftarrow{\mathbf{h_i}} ; \overrightarrow{\mathbf{h_i}} \right]$$


\subsection{CRF Layer}
It is almost always beneficial to consider the correlations between two between current label and neighboring labels since there are many syntactical constrains in natural language sentences. 
For example, I-PERSON will never follow a B-GROUP. 
If we simply feed the above mentioned hidden states independently to a Softmax layer to predict the labels, then such constrains will not be more likely to be broken. 
Linear-chain Conditional Random Field is the most popular way to control the structure prediction and its basic idea is to use a series of potential function to approximate the conditional probability of the output label sequence given the input word sequence. 

Formally, we take the above sequence of hidden states  $ \mathbf{h} = ( \mathbf{h_1}, \mathbf{h_2},...,\mathbf{h_n})$ as our input to the CRF layer, and its output is our final prediction label sequence $\mathbf{y} = ( {y_1}, {y_2},...,{y_n})$, where $y_i$ is in the set of all possible labels. 
We denote $\mathcal{Y}(\mathbf{h})$ as the set of all possible label sequences.
Then we derive the conditional probability of the output sequence given the input hidden state sequence is 

{\footnotesize $$ p(\mathbf{y}|\mathbf{h}; \mathbf{W},\mathbf{b}) 
= \frac{\prod_{i=1}^n \exp(\mathbf{W}^T_{y_{i-1},y_{i}}\mathbf{h} + \mathbf{b}_{y_{i-1},y_{i}})}
{ \sum_{\mathbf{y'} \in \mathcal{Y}(\mathbf{h})} \prod_{i=1}^n \exp(\mathbf{W}^T_{y'_{i-1},y'_{i}}\mathbf{h} + \mathbf{b}_{y'_{i-1},y'_{i}})} 
$$
}

, where $\mathbf W$ and $\mathbf b$ are the two weight matrices and the subscription indicates that we extract the weight vector for the given label pair $(y_{i-1},y_i)$.

To train the CRF layer, we use the classic maximum conditional likelihood estimation to train our model. 
The final log-likelihood with respect to the weight matrices is 

$$ L(\mathbf{W},\mathbf{b}) = \sum_{(\mathbf{h_i}, \mathbf{y_i})}  \log p(\mathbf{y_i}|\mathbf{h_i}; \mathbf{W},\mathbf{b}) $$


Finally, we adopt the Viterbi algorithm for training the CRF layer and the decoding the optimal output sequence $\mathbf{y^*}$.








\section{Experiments}
\label{sec:eval}
\subsection{Datesets}
In this shared task, the 
\subsection{Parameter Initialization}
For word-level word representation (i.e. the lookup table), 
we utilize the pretrained word embeddings from glove\footnote{\url{http://nlp.stanford.edu/data/glove.twitter.27B.zip}}.
For all out-of-vocabulary words, we assign their embeddings by randomly sampling from range $\left[-\sqrt{\frac{3}{\text{dim}}}, +\sqrt{\frac{3}{\text{dim}}}~\right]$, where \textit{dim} is the dimension of word embeddings, suggested by He et al.(\citeyear{DBLP:conf/iccv/HeZRS15}). The random initialization of character embeddings are in the same way.
We randomly initialize the weight matrices $\mathbf{W}$ and $\mathbf{b}$ with uniform samples from 
$\left[-\sqrt{\frac{6}{r+c}}, +\sqrt{\frac{6}{r+c}}~\right]$, 
$r$ and $c$ are the number of the rows and columns, following Glorot and Bengio(\citeyear{DBLP:journals/jmlr/GlorotB10}), and all LSTM hidden states are initialized to be zero except for the bias for the forget gate is initialized to be 1.0 , following Jozefowicz et al.(\citeyear{DBLP:conf/icml/JozefowiczZS15}) 


\subsection{Hyper Parameter Tuning}
We tuned the dimension of word-level embeddings from \{50, \textbf{100}, 200\}, character embeddings from \{10, \textbf{25}, 50\}, character BiLSTM hidden states (i.e. the character level word representation)  from  \{20, \textbf{50}, 100\}. 
We finally choose the bold ones.
The dimension of part-of-speech tags, dependecny roles, word positions and head positions are all 5.

As for learning method, we compare the traditional SGD and Adam\cite{}.
We found that Adam performs always better than SGD, and we tune the learning rate form \{1e-2,\textbf{1e-3},1e-4\}.

\subsection{Results} 
In comparison with other participants, the results are shown in~\tabref{tbl:compare}.

\begin{table}[th]
	\small
	\centering
	\caption{Result comparison}
	\label{tbl:compare}
	\begin{tabular}{|l|l|l|}
		\hline
		Team                & F1 (entity)    & F1 (surface form) \\ \hline
		MIC-CIS             & 36.90          & 50.38             \\ \hline
		Arcada              & 40.09          & 56.60             \\ \hline
		Drexel-CCI          & 26.81          & 59.92             \\ \hline
		\textbf{SJTU-Adapt} & \textbf{41.22} & \textbf{60.00}    \\ \hline
		FLYTXT              & 38.10          & 57.64             \\ \hline
		SpinningBytes       & 41.76          & 57.98             \\ \hline
		UH-RiTUAL           & 41.90          & 66.59             \\ \hline
	\end{tabular}
\end{table}


\section{Related Work}
\label{sec:related}
Conditional random field (CRF) is a most effective approaches 
%(Lafferty et al., \citeyear{DBLP:conf/icml/LaffertyMP01}; McCallum
%and Li, \citeyear{DBLP:conf/conll/McCallum003}; Settles, \citeyear{settles2004biomedical}) 
\cite{DBLP:conf/icml/LaffertyMP01,DBLP:conf/conll/McCallum003}
for NER and other sequence labeling tasks and it achieved the state-of-the-art performance previously in Twitter NER  \cite{baldwin2015shared}. 
Whereas, it often needs lots of hand-craft features.
More recently, Huang et al. (\citeyear{DBLP:journals/corr/HuangXY15}) introduced a similar but more complex model based on BiLSTM, which also considers hand-crafted features. 
Lample et al. (\citeyear{DBLP:conf/naacl/LampleBSKD16}) further introduced using BiLSTM to incorporate character-level word representation. Whereas, Ma and Hovy (\citeyear{DBLP:conf/acl/MaH16}) replace the BiLSTM to CNN to build the character-level word representation. Limsopatham and Collier (\citeyear{Limsopatham2016BidirectionalLF}), used similar model and achieved the best performance in the last shared task \cite{Strauss2016ResultsOT}. 
Based on the previous work, our system take more syntactical information into account, such as part-of-speech tags, dependency roles, token positions and head positions, which are proven to be effective.
 



\section{Conclusion}
In this paper, we present a novel multi-channel BiLSTM-CRF model for emerging named entity recognition in social media messages. 
We find that BiLST-CRF architecture with our proposed comprehensive word representations built from multiple information are effective to overcome the noisy and short nature of social media messages. 


\bibliographystyle{emnlp_natbib}
\bibliography{emnlp2017}

\end{document}

