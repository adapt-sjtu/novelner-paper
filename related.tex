\section{Related Work}
\label{sec:related}
Conditional random field (CRF) is a most effective approaches (Lafferty et al., 2001; McCallum
and Li, 2003; Settles, 2004) for NER and other sequence labeling tasks and it achieved the state-of-the-art performance previously in Twitter NER (Baldwin et al., 2015). 
Whereas, it often needs lots of hand-craft features.
More recently, Huang et al. (2015) introduced a similar but more complex model based on BiLSTM, which also considers hand-crafted features. 
Lample et al. (2016) further introduced using BiLSTM to incorporate character-level word representation. Whereas, Ma and Hovy (2016) replace the BiLSTM to CNN to build the character-level word representation. Nut and asd, used similar model and achieved the best performance in the last shared task (W-NUT 2016). 
Based on the previous work, our system take more syntactical information into account, such as part-of-speech tags, dependency roles, token positions and head positions, which are proven to be effective.
 


